%% PNAStwoS.tex
%% Sample file to use for PNAS articles prepared in LaTeX
%% For two column PNAS articles
%% Version1: Apr 15, 2008
%% Version2: Oct 04, 2013

%% BASIC CLASS FILE
\documentclass{pnastwo}\usepackage[]{graphicx}\usepackage[]{color}
%% maxwidth is the original width if it is less than linewidth
%% otherwise use linewidth (to make sure the graphics do not exceed the margin)
\makeatletter
\def\maxwidth{ %
  \ifdim\Gin@nat@width>\linewidth
    \linewidth
  \else
    \Gin@nat@width
  \fi
}
\makeatother

\definecolor{fgcolor}{rgb}{0.345, 0.345, 0.345}
\newcommand{\hlnum}[1]{\textcolor[rgb]{0.686,0.059,0.569}{#1}}%
\newcommand{\hlstr}[1]{\textcolor[rgb]{0.192,0.494,0.8}{#1}}%
\newcommand{\hlcom}[1]{\textcolor[rgb]{0.678,0.584,0.686}{\textit{#1}}}%
\newcommand{\hlopt}[1]{\textcolor[rgb]{0,0,0}{#1}}%
\newcommand{\hlstd}[1]{\textcolor[rgb]{0.345,0.345,0.345}{#1}}%
\newcommand{\hlkwa}[1]{\textcolor[rgb]{0.161,0.373,0.58}{\textbf{#1}}}%
\newcommand{\hlkwb}[1]{\textcolor[rgb]{0.69,0.353,0.396}{#1}}%
\newcommand{\hlkwc}[1]{\textcolor[rgb]{0.333,0.667,0.333}{#1}}%
\newcommand{\hlkwd}[1]{\textcolor[rgb]{0.737,0.353,0.396}{\textbf{#1}}}%

\usepackage{framed}
\makeatletter
\newenvironment{kframe}{%
 \def\at@end@of@kframe{}%
 \ifinner\ifhmode%
  \def\at@end@of@kframe{\end{minipage}}%
  \begin{minipage}{\columnwidth}%
 \fi\fi%
 \def\FrameCommand##1{\hskip\@totalleftmargin \hskip-\fboxsep
 \colorbox{shadecolor}{##1}\hskip-\fboxsep
     % There is no \\@totalrightmargin, so:
     \hskip-\linewidth \hskip-\@totalleftmargin \hskip\columnwidth}%
 \MakeFramed {\advance\hsize-\width
   \@totalleftmargin\z@ \linewidth\hsize
   \@setminipage}}%
 {\par\unskip\endMakeFramed%
 \at@end@of@kframe}
\makeatother

\definecolor{shadecolor}{rgb}{.97, .97, .97}
\definecolor{messagecolor}{rgb}{0, 0, 0}
\definecolor{warningcolor}{rgb}{1, 0, 1}
\definecolor{errorcolor}{rgb}{1, 0, 0}
\newenvironment{knitrout}{}{} % an empty environment to be redefined in TeX

\usepackage{alltt}

%% ADDITIONAL OPTIONAL STYLE FILES Font specification

%\usepackage{pnastwoF}



%% OPTIONAL MACRO DEFINITIONS
\def\s{\sigma}
%%%%%%%%%%%%
%% For PNAS Only:
\url{www.pnas.org/cgi/doi/10.1073/pnas.0709640104}
\copyrightyear{2008}
\issuedate{Issue Date}
\volume{Volume}
\issuenumber{Issue Number}
%\setcounter{page}{2687} %Set page number here if desired
%%%%%%%%%%%%

\usepackage[latin1]{inputenc}
\usepackage[english]{babel}
\IfFileExists{upquote.sty}{\usepackage{upquote}}{}

\begin{document}

\title{Hurricanes: the female of the species in no more deadly than the male}
\author{R.B. O'Hara\affil{1}{Biodiversity and Climte CHange Research Centre, BiK-F, Frankfurt am Main, Germany}}
\contributor{Submitted to Proceedings of the National Academy of Sciences
of the United States of America}

\maketitle

\begin{article}
\begin{abstract}
{A re-analysis of the data on damage, death toll, strength, and name of hurricanes provided by Jung {\em  et al.} \cite{1} suggests that feminine-names hurricanes do not cause more deaths, because the effects of damage had been mis-specified.}
\end{abstract}

\keywords{hurricanes | gender | residuals | plotting}

\abbreviations{AIC, Akaike's Information Criterion}




Jung {\em et al.} \cite{1} analysed data on hurricanes and concluded that if hurricanes were more powerful, they were more deadly when the names were more feminine. I suggest that this was an artifact of an incorrect analysis, and a standard exploration of the fit of the model shows this. After the authors' model is fitted, an examination of the residuals suggests reveals structure in the effect of normalized damage: rather than the residuals being an unstructured cloud, thes show a distinctly curved shape (Figure~\ref{fig:residplots}a).\\

If we adjust the model by adding the square root of damage as a term, this structure is no longer present (Figure~\ref{fig:residplots}b). The moel fit is improved (AIC for original model: 660.7, AIC for model with square root term: 634.9) In addition, name feminity no longer has an effect: removing the effects of feminity leads to a more parsimonious model (AIC: 630.1).\\

From a statistical perspective, classical model checking appraoches lead to a better model. Whether it is more likely that the effects of damage are non-linear, or whether they are linear and an effect of hurricane name is one where expert knowledge (of those studying responses to disaster) should be called upon.

\begin{acknowledgments}
More details about the analysis are provided at http://rpubs.com/oharar/19171. This study was funded by the ''Landesoffensive zur Entwicklung wissenschaftlich-\"{o}konomischer Exzellenz'' (LOEWE) of the state Hesse in Germany through the Biodiversity and Climate Research Centre (Bik-F).\end{acknowledgments}

\begin{thebibliography}{10}
\bibitem{1}
K.~Jung, S.~Shavitt, M.~Viswanathan, J.M.~Hilbe, {\em Female hurricanes are deadlier than male hurricanes}, Proceedings of the National Academy of Sciences, early edition (2014). DOI: 10.1073/pnas.1402786111

\end{thebibliography}

\end{article}

\begin{figure}
\caption{Plots of deviance residuals against squar root of normalised damage for (a) model used in \cite{1}, (b) model with square root of normalised damage, and its interaction with perceived masculinity-feminity of hurricane names}
\label{fig:residplots}
\begin{knitrout}
\definecolor{shadecolor}{rgb}{0.969, 0.969, 0.969}\color{fgcolor}

{\centering \includegraphics[width=8cm,height=8cm]{figure/unnamed-chunk-1} 

}



\end{knitrout}

\end{figure}

\end{document}
