%% PNAStwoS.tex
%% Sample file to use for PNAS articles prepared in LaTeX
%% For two column PNAS articles
%% Version1: Apr 15, 2008
%% Version2: Oct 04, 2013

%% BASIC CLASS FILE
\documentclass{pnastwo}

%% ADDITIONAL OPTIONAL STYLE FILES Font specification

%\usepackage{pnastwoF}



%% OPTIONAL MACRO DEFINITIONS
\def\s{\sigma}
%%%%%%%%%%%%
%% For PNAS Only:
\url{www.pnas.org/cgi/doi/10.1073/pnas.0709640104}
\copyrightyear{2008}
\issuedate{Issue Date}
\volume{Volume}
\issuenumber{Issue Number}
%\setcounter{page}{2687} %Set page number here if desired
%%%%%%%%%%%%

\usepackage[latin1]{inputenc}
\usepackage[english]{babel}

\begin{document}

\title{Hurricanes: the female of the species in no more deadly than the male}
\author{R.B. O'Hara\affil{1}{Biodiversity and Climte CHange Research Centre, BiK-F, Frankfurt am Main, Germany}}
\contributor{Submitted to Proceedings of the National Academy of Sciences
of the United States of America}

\maketitle

\SweaveOpts{concordance=TRUE}

\begin{article}
\begin{abstract}
{A re-analysis of the data on damage, death toll, strength, and name of hurricanes provided by Jung \emph{et al.} \cite{Jungetal}}
\end{abstract}

\keywords{monolayer | structure | x-ray reflectivity | molecular electronics}

\abbreviations{SAM, self-assembled monolayer; OTS, octadecyltrichlorosilane}

Jung \emph{et al.} 



\begin{thebibliography}{10}
\bibitem{Jungetal}
K.~Jung, S.~Shavitt, M.~Viswanathan, J.M.~Hilbe, {\em Female hurricanes are deadlier than male hurricanes}, Proceedings of the National Academy of Sciences, early edition (2014). DOI: 10.1073/pnas.1402786111

\end{thebibliography}


\end{document}